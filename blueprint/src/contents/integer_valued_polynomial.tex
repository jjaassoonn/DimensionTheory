\section{Integer Valued Polynomials}
\label{sec:integer_valued_polynomial}

\begin{definition}[Binomial Polynomial]
    \label{def:binomial_polynomial}
    \lean{binomialPolynomial}
    \leanok
    For any field $F$ with characteristic $0$, we define the binomial polynomials in $F[X]$ as
    \[
      \left\{ Q_k(X) = {X \choose k} \mid k \in \mathbb{N} \right\}.
    \]
    \begin{equation}
        \begin{aligned}
            Q_0(X) &= 1 \\
            Q_1(X) &= X \\
            Q_2(X) &= \frac{X(X-1)}{2} \\
            Q_3(X) &= \frac{X(X-1)(X-2)}{6} \\
            Q_4(X) &= \frac{X(X-1)(X-2)(X-3)}{24} \\
            &\vdots
        \end{aligned}
    \end{equation}
\end{definition}

\begin{lemma}
    For any field $F$ with characteristic $0$, the binomial polynomials forms a basis for $F[X]$.
    \lean{binomialPolynomial.basis}
    \leanok
    \uses{def:binomial_polynomial}
\end{lemma}
\begin{proof}
    The set of binomial polynomials is linearly independent since they have distinct degrees.
    % TODO
\end{proof}

%%% Local Variables:
%%% mode: latex
%%% TeX-master: "introduction"
%%% TeX-master: "../web"
%%% End:
